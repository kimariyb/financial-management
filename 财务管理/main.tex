% !TEX program = xelatex
\documentclass[
  10pt,
  twoside,
  openany,
  b5paper, % 以上均为 ctexbook 提供的文类选项
  colorscheme = black, % 请根据需要选择或定制配色方案
  xits = false,
]{qyxf-book}


\title{财务管理绝艺总纲威力加强版}
\subtitle{}  % 可选
\author{喵小决KIMARIYB}
\date{2023 年 2 月 7 日}
\typo{喵小决KIMARIYB}
%\typo{AlphaGo}  % 排版人员信息,选填

% 定制元信息
\org{\Large\textit{江西理工大学}\\\textsc{Jiangxi University Of Science And Technology}}
\footorg{\textsc{Miao Xiao Jue}}
\license{}  % 清空许可证信息

\cover{\includegraphics[width=0.6\textwidth]%
	{figure/Xiaohui.pdf}}

% 调整封面标题大小
\renewcommand{\titlefont}{\Huge\bfseries}
\renewcommand{\subtitlefont}{\LARGE\itshape}

% 取消本条注释,则将定义计数器归并到定理下
% \resetdefine
\usepackage{booktabs}
\usepackage{amsmath}
\usepackage{newtxtext}
\usepackage{newtxmath}
\usepackage{multirow}
\usepackage{fontawesome5}

\begin{document}

\maketitle
\tableofcontents

\chapter*{前言}

为了解决期末考试问题,本人编写了这一份财务管理学习指南,全书采用\LaTeX 整理,并使用了西安交通大学钱院学辅的\LaTeX 样式。本书分为三个章节:考试的重点、必备的公式以及模拟试题三部分,全由本人编写并排版。编写这本书的目的,是希望在期末考试来临之际,能给学习这门课程的同学带来帮助。

本书的内容参考自江西理工大学中的软件金融191班、财务管理201班、财务管理202班、财管203班以及财务管理204班的同学在2022-2023学年上学期的财务管理这一学科的学习中的学习精华。特向给本人提供这些内容的同学表示感谢。

本书适用于江西理工大学南昌校区,在2022-2023学年上学期学习财务管理这一门学科的同学。由于每年的考题都不一样,因此本书具有时效性,特别是考试的重点一章节,如果你不是上述涉及到的同学,可以选择不浏览。但是我强烈建议您浏览并完成模拟考试一章节,题目简单,通俗易懂,很适合用来练手,为您期末考试过关打下充足的基础。

使用者可以在给出作者署名及资料来源
的前提下对本作品进行转载,但不得对本作品进行修改,亦不得基于本
作品进行二次创作,不得将本作品运用于商业用途。本作品已经发布于Github上,发布地址为:

https://github.com/kimariyb/financial-management

由于本人并不是来自财务管理或会计专业,在书中难免会有部分遗漏和错误,需要各位使用者能帮助指正。如您在参考的过程中发现有任何错误
之处,欢迎您通过下面的方式联系我,帮助我们改进:
\begin{itemize}
	\item \faGithub ~~ 本人Github:~~https://github.com/kimariyb
	\item \faQq ~~ 本人QQ:~~2420707848
\end{itemize}



\begin{flushright}
	喵小决 Kimariyb \\
	2023 年 2 月 7 日
\end{flushright}

\chapter{考试の重点}

\section{简答题}
\exercise{1} \textbf{公司价值最大化的优点和缺点是什么?}
\begin{note}
	\textbf{公司价值最大化的优点:}
	\begin{itemize}
		\item 考虑了货币的时间价值和投资的风险价值;
		\item 反映了对企业资产保值、增值的要求;
		\item 有利于克服企业的短期行为;
		\item 有利于社会资源的合理配置,实现社会效益最大化。
	\end{itemize}
	\textbf{公司价值最大化的缺点:}
	\begin{itemize}
		\item 对非上市公司很难适用;
		\item 企业的价值,特别是股票的价值不一定为企业所控制,其价格波动也并非与企业财务状况相一致。
	\end{itemize}
	
\end{note}



\exercise{2} \textbf{财务目标有哪些观点,他们的优缺点分别是什么?}

\begin{note}
	财务目标主要有利润最大化和股东财富最大化两种。
	
	利润最大化认为利润代表了企业新创造的财富,利润越多则企业的财富越多,越接近企业的目标。\textbf{利润最大化的缺陷:}
	\begin{itemize}
		\item 没有考虑到货币的时间价值;
		\item 没有反映创造的利润与投入的资本之间的关系;
		\item 没有考虑到风险的因素;
		\item 可能会导致公司的短期行为;
		\item 不能反映企业未来的盈利能力;
		\item 可能反映不了企业的真实情况。
	\end{itemize}
	
	股东财富最大化是指通过财务上的合理运营,为股东创造更多的财富。\textbf{股东财富最大化的优点:}
	\begin{itemize}
		\item 能够考虑到取得收益的时间因素和风险因素;
		\item 能够克服企业在追求利润上的短期行为;
		\item 能够充分体现企业所有者对资本保值与增值的要求。
	\end{itemize}
	\textbf{股东财富最大化的缺点:}
	\begin{itemize}
		\item 只适用于上市公司,对非上市公司则很难适用;
		\item 股票价值的波动是受多种因素影响的综合结果,它的高低实际上不能够完全反映股东财富或价值的大小。
	\end{itemize}
\end{note}

\exercise{3} \textbf{财务分析有哪些作用与目的?一般财务比率分析主要从哪些方面进行分析?}

\begin{note}
	\textbf{财务分析的作用:}
	\begin{itemize}
		\item 通过财务分析,可以全面评价企业在一定时期的各种财务能力,从而分析企业的问题。
		\item 通过财务分析,可以为企业外部人员提供更完整的会计信息,便于其了解企业的财务状况。
		\item 通过财务分析,可以检查企业内部各部门完成计划的情况,考核各部门业绩,有利于保证企业财务目标的实现。
	\end{itemize}
	
	\textbf{财务分析的目的:}
	\begin{itemize}
		\item 分析企业控制现金流量的能力和在多变的经济环境下保持稳定下的财务基础的能力。
		\item 分析企业的盈利能力和风险状况,以据此评估股票价值,进行有效的投资决策。
		\item 通过财务分析所提供的信息来监控企业的经营活动和财务状况,以便更早的发现问题。
		\item 更好地了解宏观经济的运行情况和企业的经营活动是否合法,以便制定相关政策。
	\end{itemize}
	
	\textbf{财务比率分析主要从以下方面进行分析:}
	\begin{itemize}
		\item 偿债能力分析:短期偿债能力分析可以通过分析的比率有流动比率、速动比率、现金流量比率等。长期偿债能力分析的比率有资产负债率、股东权益比率、负债与股东权益比率和利息保障倍数等。
		\item 营运能力分析:常见的比率有应收账款周转率、存货周转率、流动资产周转率、固定资产周转率、总资产周转率等。
		\item 盈利能力分析:常见的比率有资产报酬率、股东权益报酬率、销售毛利率和销售净利率、成本费用净利率等。
		\item 发展能力分析:常见的比率有销售增长率、资产增长率、股权资本增长率、利润增长率等。
	\end{itemize}
\end{note}

\exercise{4} \textbf{简述财务比率在公司财务分析中的重要作用?}
\begin{note}
	通过财务比率分析,可以对企业的偿债能力、资产运用能力、盈利能力和市场价值有所了解,进而对企业的总体财务状况和未来发展趋势做出一定的预测和判断。
\end{note}

\exercise{5} \textbf{债券价值的影响因素及其对债券价值影响的主要表现是什么?}

\begin{note}
	\begin{itemize}
		\item 债券面额。债券面额是决定债券价格的最基本因素。债券面额越大,债券价值越大。
		\item 票面利率。票面利率是债券的名义利率。票面利率越大,债券价值越大。
		\item 市场利率。债券发行时的市场利率越高,债券价值越低。
		\item 债券期限。债券的期限越长,债权人的风险越大,债券价值越低。
	\end{itemize}
\end{note}

\exercise{6} \textbf{什么是资本成本,资本成本的作用是什么?}
\begin{note}
	资本成本时企业筹集和使用资本的代价,如筹资公司向银行支付的借款利息等。资本成本也可以是投资者要求的必要报酬和最低报酬。
	
	\textbf{资本成本的作用:}
	\begin{itemize}
		\item 资本成本时选择筹资方式、进行资本结构决策和选择追加筹资方案的依据。
		\item 资本成本时评价投资项目、比较投资方案和进行投资决策的经济标准。
		\item 资本成本可以作为评价企业整体经营业绩的基准。
	\end{itemize}
\end{note}

\exercise{7} \textbf{什么是股利政策?实践中有哪些股利政策,谈谈你对他们的认识?}
\begin{note}
	股利政策是指确定公司的净利润如何分配的方针和策略的总称。股利政策有剩余股利政策、固定或稳定增长的股利政策、固定股利支付率政策、低正常股利加额外股利政策。
	\begin{itemize}
		\item 剩余股利政策是一种投资优先的股利政策。采取这种股利政策的先决条件是公司具有良好的投资机会,并且投资机会的预期报酬率要高于股东要求的必要报酬率。
		\item 固定股利政策是指公司在较长时期内每股支付固定股利额的股利政策。其能够很好的使股利保持稳定的水平。
		\item 稳定型增长股利政策使指在一定的时期内保持公司的每股股利额稳定增长的股利政策。使用这种股利政策的公司一般会随着公司盈利的增加,保持每股股利平稳提高。
		\item 固定股利支付率政策是一种变动的股利政策,公司每年都从净利润种按固定的股利支付率发放现金股利。
		\item 低正常股利加额外股利政策是一种介于固定股利政策和变动股利政策之间的折中的股利政策。
	\end{itemize}
\end{note}

\exercise{8} \textbf{MM理论的基本假设是什么?}
\begin{note}
	\begin{itemize}
		\item 公司在无税收的环境中经营;
		\item 公司营业风险的高低由息税前利润标准差来衡量;公司营业风险决定其风险等级;
		\item 投资者对所有公司未来盈利及风险的预期相同;
		\item 投资者不支付证券交易成本,所有债券利率相同;
		\item 公司为零增长公司,即年平均盈利额不变;
		\item 个人和公司均可发行无风险债券,并有无风险利率;
		\item 公司无破产成本;
		\item 公司的股利政策与公司价值无关,公司发行新债时不会影响已有债券的市场价值;
		\item 存在高度完善和均衡的资本市场。
	\end{itemize}
\end{note}

\exercise{9} \textbf{简述债权资本在资本结构中的作用?}
\begin{note}
	\begin{itemize}
		\item 合理安排债务资本比例可以降低企业的综合资本率。
		\item 合理安排债务资本比例可以获得财务杠杆利益。
		\item 合理安排债务资本比例可以增加公司的价值。
	\end{itemize}
\end{note}

\newpage

\exercise{10} \textbf{简述企业筹资组合策略有哪些?}
\begin{note}
	\begin{itemize}
		\item 正常的筹资组合:短期资产由短期资金来融通,长期资产由长期资金来融通。
		\item 冒险的筹资组合:将部分长期资产由短期资金来融通。该种策略的资金成本较低,因而能减少利息支出,增加企业收益。但用短期资金融通了一部分长期资产,风险比较大。
		\item 保守的筹资组合:将部分短期资产用长期资金来融通,采用该种筹资方式,短期资产的一部分和全部长期资产都用长期资金来融通,而只有一部分短期资产用短期资金来融通。该种策略的风险较小,但成本较高,会使企业的利润减少。
	\end{itemize}
\end{note}


\section{名词解释}
\begin{enumerate}
	\item \textbf{半强势效率性}:证券市场中,债券价格不仅包括过去的价值信息,还包括所有公开的其他信息。
	\item \textbf{货币的时间价值}:是指在不考虑风险和通货膨胀的情况下,货币经过一定时间的投资和再投资所产生的增值。
	\item \textbf{普通年金}:又称为后付年金,是指从第一期起,在连续间隔期相等的多个时期中,每期期末收付等额年金的年金。
	\item \textbf{先付年金}:又称先付年金、即付本金或期初年金,是指从第一期起,在连续间隔期相等的多个时期中,每期期初等额收付的系列款项。
	\item \textbf{系统性风险}:是指那些影响市场上所有公司的因素导致的风险,不能够被抵消,系统性风险的程度用$\beta$系数来衡量。
	\item \textbf{非系统性风险}:是由于某些随机事件导致的风险,这种风险可以被证券持有的多样化抵消。
	\item \textbf{证券市场线}:资本资产定价模型的图示形式称为证券市场线。它主要用来说明投资组合报酬率与系统风险程度$\beta$系数之间的关系,以及市场上所有风险性资产的均衡期望收益率与风险之间的关系。
	\item \textbf{无风险报酬率}:是指把资金投资于一个没有任何风险的投资对象所能得到的收益率。一般会把这一收益率作为基本收益,再考虑可能出现的各种风险。一般用政府公债的利息率来衡量。
	\item \textbf{必要收益率}:又叫最低必要报酬率或最低要求的收益率,表示投资者对某资产合理要求的最低收益率。
	\item \textbf{历史标准}:根据过去的实际的历史资料而产生的连续的财务比率标准。
	\item \textbf{比率分析法}:是将企业同一时期的财务报表中的相关项目进行对比,得出一系列的财务比率,以此来揭示企业财务状况的分析方法。
	\item \textbf{杜邦分析法}:是利用几种主要的财务比率之间的关系来综合地分析企业的财务状况。具体来说,它是一种用来评价公司盈利能力和股东权益回报水平,从财务角度评价企业绩效的一种经典方法。其基本思想是将企业净资产收益率逐级分解为多项财务比率乘积,这样有助于深入分析比较企业经营业绩。
	\item \textbf{市盈率}:是指股票价格除以每股收益的比率。或以公司市值除以年度股东应占溢利。
	\item \textbf{权益乘数}:股东权益比例的倒数称为权益乘数,即资产总额是股东权益总额的多少倍。权益乘数反映了企业财务杠杆的大小,权益乘数越大,说明股东投入的资本在资产中所占的比重越小,财务杠杆越大。
	\item \textbf{托宾Q指标}:是企业市场价值对其资产重置成本的比率。反映的是一个企业两种不同价值估计的比值。
	\item \textbf{普通股}:公司发行的代表股东享有平等权利、义务、不加特别限制、股利不固定的股票。
	\item \textbf{优先股}:公司发行的优先于普通股股东分取股利和公司剩余财产的股票。
	\item \textbf{短期租赁}:又被称为经营租赁,是由出租人向承担企业提供租赁设备,并提供设备维修保养和人员培训等的服务性业务。
	\item \textbf{获利指数}:是指投产后按基准收益率或设定折现率折算的各年现金流入量的现值合计与原始投资的现值合计之比。
	\item \textbf{内含报酬率}:是指能够使未来现金流入量现值等于未来现金流出量现值的折现率,或者说是使投资方案净现值为零的折现率。该指标反映了投资项目的真实报酬。
	\item \textbf{净现值}:投资项目投入使用后的净现金流量,按资本成本率或企业要求达到的报酬率折算为现值,减去初始投资以后的余额。
	\item \textbf{营运资金}:是合营企业流动资产总额减流动负债总额后的净额,即企业在经营中可供运用、周转的流动资金净额。
	\item \textbf{财务风险}:是指企业经营活动中与筹资有关的风险,尤其是指在筹资活动中利用财务杠杆可能导致企业股权资本所有者收益下降的风险。
	\item \textbf{经营风险}:是指与企业经营有关的风险,尤其是指企业在经营活动中利用营业杠杆而导致息税前利润下降的风险。
	\item \textbf{剩余股利政策}:剩余股利政策是企业在有良好的投资机会时,根据目标资本结构测算出必须的权益资本与既有权益资本的差额,首先将税后利润满足权益资本需要,而后将剩余部分作为股利发放的政策。
	\item \textbf{股票分割}:是指将额外的股份按现有持股比例分配给各股东。
\end{enumerate}

\chapter{必备の公式}
\section{有关第二章的计算}
\subsection{复利终值和复利现值}
复利终值的计算公式为:
\begin{equation*}
	F_n=P(1+i)^n
\end{equation*}
式中,$F_n$表示复利终值;$P$表示复利现值;$i$表示利息率;$n$表示计息期数。式中的$(1+i)^n$称为复利终值系数,简写成$(F/P,i,n)$

复利现值的计算公式可以由终值的计算公式导出:
\begin{equation*}
	P = \frac{F_n}{(1+i)^n} = F_n \cdot \frac{1}{(1+i)^n} 
\end{equation*}
式中的$\dfrac{1}{(1+i)^n}$称为复利现值系数或者折现系数,用$(P/F,i,n)$表示。

\subsection{年金终值和现值}
\textbf{1. 普通年金终值和现值}

普通年金是指每期期末有等额收付款项的年金。其第一次支付是在第一期的期末,其终值是在最后一次支付的时点。

普通年金终值的计算公式为
\begin{equation*}
	F_n = A \sum_{t=1}^{n} (1+i)^{t-1} 
\end{equation*}
式中,$A$表示年金数额,$i$代表利息率,$n$表示计息期数,$F_n$表示年金终值,$\sum\limits_{t=1}^{n}(1+i)^{t-1}$称为年金终值系数,简写为$(F/A,i,n)$。年金终值系数还可以这样求出:
\begin{equation*}
	(F/A,i,n) = \frac{(1+i)^n -1}{i}
\end{equation*}

普通年金现值的计算公式为:
\begin{equation*}
	P_n = A\sum_{t=1}^{n} \frac{1}{(1+i)^t} 
\end{equation*}
式中,$\sum\limits_{t=1}^{n} \dfrac{1}{(1+i)^n}$称为年金现值系数,可简写为$(P/A,i,n)$。除了上式以外,年金现值系数还有另外一个公式:
\begin{equation*}
	(P/A,i,n) = \frac{(1+i)^n-1}{i(1+i)^n}
\end{equation*}

\textbf{2. 先付年金终值和现值}

先付年金终值的公式可以由普通年金的公式导出:
\begin{equation*}
	F_n = A \cdot (F/A,i,n) \cdot (1+i) =A \cdot (F/A,i,n+1)-A
\end{equation*}

先付年金现值的公式为:
\begin{equation*}
	P_n = A \cdot (P/A,i,n) \cdot (1+i)=A\cdot (P/A,i,n-1) + A
\end{equation*}

\textbf{3. 延期年金现值的计算}

延期年金的终值计算与普通年金终值的计算公式相同,其现值的计算可以先计算普通年金的现值,后计算复利现值,其公式为:
\begin{equation*}
	P_0 = A \cdot (P/A,i,n) \cdot (P/F,i,m)
\end{equation*}
式中的$P_0$表示延期年金的现值。除此之外还可以这样计算:
\begin{equation*}
	P_0 = A\cdot (P/A,i,m+n) - A\cdot (P/A,i,m)
\end{equation*}

\textbf{4. 永续年金现值的计算}

永续年金没有终值,其现值为:
\begin{equation*}
	P_0 = A \cdot \frac{1}{i}
\end{equation*}

\begin{note}
	实际利率的计算方法:
	\begin{equation*}
		i = \left(1+\frac{r}{m}\right)^m-1
	\end{equation*}
式中,$i$表示实际利率,$r$表示名义利率,$m$表示计息期次数。
\end{note}

\newpage

\subsection{风险和报酬}

\emph{
有一企业有}A、B\emph{两个投资项目,两个投资项目的收益率及其概率分布情况如下表所示:}

\begin{table}[htbp]
	\centering
	\begin{tabular}{ccccc}
		\toprule
		\textbf{项目实施情况} & \multicolumn{2}{c}{\textbf{该种情况出现的概率}} & \multicolumn{2}{c}{\textbf{投资收益率}} \\
		\cmidrule{2-5}
		& 项目A & 项目B & 项目A & 项目B \\
		\midrule
		好 & 0.2 & 0.3 & 15\% & 20\% \\
		一般 & 0.6 & 0.4 & 10\% & 15\% \\
		差 & 0.2 & 0.3 & 0 & -10\% \\
		\bottomrule
	\end{tabular}
\end{table}

\textbf{1. 期望报酬率的计算}

如果计算项目A的期望报酬率$\overline{R}_\mathrm{A}$,可以用公式:
\begin{equation*}
	\begin{aligned}
		\overline{R}_\mathrm{A} & = P_1R_1 + P_2R_2 + \cdots + P_iR_i = \sum_{i=1}^{n} P_iR_i \\
					 & = 0.2 \times 15\% + 0.6 \times 10\% + 0.2 \times 0 \\
					 & = 0.09 =9\%
	\end{aligned}
\end{equation*}
式中的$P_i$表示情况出现的概率,$R_i$表示收益率。如果是证券组合的期望报酬率,则将概率$P_i$改为比重$\omega_i$。

\textbf{2. 标准差和方差的计算}

如果计算项目A的方差$\sigma^2_\mathrm{A}$,可以用公式:
\begin{equation*}
	\begin{aligned}
		\sigma^2_\mathrm{A} &= \sum_{i=1}^{n} (R_i-\overline{R})^2 P_i \\
				 &= (0.15-0.09)^2 \times 0.2 + (0.10-0.09)^2 \times 0.6 + (0-0.09)^2 \times 0.2 \\
				 &= 0.0024
	\end{aligned}
\end{equation*}
如果将得到的方差做算术平方根运算,则得到标准差$\sigma$。方差和标准差是衡量整体风险的量度。在证券组合计算反差时,会有相关系数$\rho$出现,其定义为(不要求掌握):
\begin{equation*}
	\rho_{XY}=\frac{\mathrm{Cov}(X,Y)}{\sigma_X \sigma_Y}	
\end{equation*}
如果当股票报酬完全负相关($\rho=-1$)时,所有的风险都能够被分散;而当股票报酬完全正相关($\rho=1$)时,风险将无法分散。

\textbf{3. 离散系数的计算}

离散系数$C$的公式为:
\begin{equation*}
	C = \frac{\sigma}{\overline{R}}
\end{equation*}
在期望值不同的情况下,离散系数越大,风险越大;离散系数越小,风险越小。

\textbf{4. $\beta$系数}

$\beta$系数是度量一项资产系统性风险的指标,不同资产的系统风险性不同,度量一项资产的系统性风险的指标是$\beta$系数。单只股票的$\beta$系数的计算不需要掌握,证券组合的$\beta$系数可以用加权的算法来计算:
\begin{equation*}
	\beta = \sum_{i=1}^{n} \omega_i \beta_i
\end{equation*}

如果$\beta=1$,则该资产的报酬率与市场平均报酬率呈同方向、同比例变化。该资产所含的系统性风险与市场组合的风险一致。

如果$\beta <1$,则该资产的报酬率的波动幅度小于市场组合报酬率的波动幅度,所含的系统性风险小于市场组合的风险。

如果$\beta >1$,则该资产的报酬率的波动幅度大于市场组合报酬率的波动幅度,所含的系统性风险大于市场组合的风险。

\textbf{5. 资本资产定价模型}

证券组合的风险报酬率$R_\mathrm{p}$的计算公式为:
\begin{equation*}
	R_\mathrm{p} = \beta_\mathrm{p} (R_\mathrm{M}-R_\mathrm{F})
\end{equation*}
式中,$R_\mathrm{M}$表示的是所有股票的平均报酬率,简称市场报酬率;$R_\mathrm{F}$表示无风险报酬率,一般用政府的公债利息率来衡量。

市场的期望报酬率是无风险资产的报酬率加上因市场组合的内在风险所需的补偿,用公式表示为:
\begin{equation*}
	R_\mathrm{M} = R_\mathrm{F} + R_\mathrm{P}
\end{equation*}
式中,$R_\mathrm{M}$表示市场的期望报酬率,$R_\mathrm{F}$表示无风险资产的报酬率,$R_\mathrm{P}$表示投资者因持有市场组合而要求的风险溢价。

资本资产模型的一般形式为:
\begin{equation*}
	R_i = R_\mathrm{F} + \beta_i (R_\mathrm{M} - R_\mathrm{F})
\end{equation*}
式中,$R_i$表示第$i$种证券组合的必要报酬率,其余符号的含义与上述相同。

\newpage

\section{有关第五章的计算}
\subsection{债券发行价值}


债券的发行价格$V$可以用下式进行计算:
\begin{equation*}
	V = \frac{B}{(1+R_\mathrm{M})^n}+\sum_{t=1}^{n}\frac{I}{(1+R_\mathrm{M})^t}
\end{equation*}
式中,$B$代表债券面额,即债券到期偿付的本金;$I$代表债券年利息,即债券面额与债券票面年利率的乘积;$R_\mathrm{M}$表示债券发售时的市场利率;$n$表示债券期限;$t$表示债券付息期数。

\subsection{租赁租金}

\emph{某企业于}20×5\emph{年}1\emph{月}1\emph{日}\emph{从租赁公司租入一套设备,价值}60\emph{万元,租期}6\emph{年,预计租赁期满时的残值为}5\emph{万元,归租赁公司,年利率}8\%\emph{,租赁手续费每年}2\%\emph{,租金每年年末支付一次,则每年租金额为多少?}

\solve[答]:先计算折现率$r$,折现率等于年利率$i$加上租赁手续费$f$:
\begin{equation*}
	r = i + f = 0.08 + 0.02 = 0.1 = 10\%
\end{equation*}
根据流入的现值等于流出的现值,初始流入了60万元,这60万元应该等于流出的资金,则可以列出下式:
\begin{equation*}
	60 = A \times (P/A, 10\%, 6) + 5 \times (P/F, 10\%, 6) 
\end{equation*}
整理得到每年末租金$A$为:
\begin{equation*}
	\begin{aligned}
		A &= \frac{60 - 5 \times(P/F, 10\%, 6)}{(P/A, 10\%, 6)} = 13.13 \text{万元} 
	\end{aligned}
\end{equation*}
得到计算租金的普通公式:
\begin{equation*}
	A = \frac{C-S(P/F,r,n)}{(P/A,r,n)}
\end{equation*}
以上方法称为等额年金法,资本成本率通常也可以作为折现率进行计算。计算租金还有另外一种方法,称为平均分摊法,其公式为:
\begin{equation*}
	A = \frac{(C-S)+I+F}{N} 
\end{equation*}
式中,$A$表示每次支付的租金,$C$表示租赁设备购置的成本,$S$表示租赁设备的预计残值,$I$表示租赁期间的利息;$F$表示租赁期间手续费,$N$表示租期。

\newpage

\section{有关第六章的计算}
\subsection{资本成本率}

\textbf{1. 债务资本成本率的计算}

(1)在不考虑货币的时间价值的情况下,资本成本率$K$的计算公式为:
\begin{equation*}
	K = \frac{D}{P-f} = \frac{D}{P(1-F)}
\end{equation*}
式中,$D$表示年资金占用费(用资费用);$P$表示筹资总额;$F$表示筹资费用率;$f$表示筹资费用。其主要适用于债务资本成本的计算。

(2)在考虑货币的实际价值的情况下,资本成本率$K$的计算公式为:
\begin{equation*}
	K = \frac{I(1-T)}{L(1-F)}
\end{equation*}
式中,$I$表示长期借款年利息额;$L$表示借款的本金;$F$表示长期借款的借款手续费率;$T$表示所得税税率。其主要适用于金额大,借贷时间长的情况。

\analysis[注]:如果不特别说明,债务资本成本就是指的是税后的债务资本成本。其与税前的债务资本成本之间的关系为:
\begin{equation*}
	K = R_\mathrm{d}(1-T)
\end{equation*}
式中,$K$表示债务资本成本率;$R_\mathrm{d}$表示税前资本成本率,也称为企业债务利息率;$T$表示企业所得税税率。

\emph{某企业取得五年期的长期借款}200\emph{万元,年利率}10\%\emph{,每年付息一次,到期一次还本,借款费用率}0.2\%\emph{,企业所得税税率}25\%\emph{,该借款的资本成本率为多少?}

\solve[答]:套用上述长期借款资本成本率的公式,得:
\begin{equation*}
	\begin{aligned}
		K &= \frac{I(1-T)}{L(1-F)} \\[1.5ex]
		&= \frac{200 \times 10\% \times (1-25\%)}{200 \times (1-0.2\%)} \\[1.5ex]
		&=7.52\%
	\end{aligned}
\end{equation*}

\textbf{2. 股权资本成本率的计算}

(1)如果要计算优先股的资本成本率,是非常简单的,优先股股息通常是固定的,其计算公式如下:
\begin{equation*}
	K = \frac{D}{P}
\end{equation*}
式中,$K$表示优先股的资本成本率;$D$表示优先股每年的股息;$P$表示优先股筹资净额,即发行价格扣除发行费用后的金额。

(2)如果要计算普通股的资本成本率就相对而言比较复杂。如果公司实行固定股利政策,每年分派现金股利相等,则资本成本率的计算同上。如果公司实行固定增值股利政策,则需要按照下式进行计算:
\begin{equation*}
	K = \frac{D_1}{P_0(1-f)}+G
\end{equation*}
式中,$K$表示资本成本率;$G$表示股利固定增长率;$D_1$表示将要支付还未支付的股利;$P_0$表示股票目前市场价格;$f$表示发行费用。

\emph{某股票普通股市价}30\emph{元,筹资费用率}2\%\emph{,本年发放现金股票股利每股}0.6元\emph{,预期股利年增长率为}10\%\emph{,计算该普通股的资本成本。}

\solve[答]:由题意知,首先计算未支付的股利$D_1$:
\begin{equation*}
	D_1 = 0.6 \times (1+10\%) = 0.66\text{元}
\end{equation*}

将已知数据带入普通股资本成本计算公式中:
\begin{equation*}
	\begin{aligned}
		K &= \frac{D_1}{P_0(1-f)} +G \\[1.5ex] &= \frac{0.66}{30 \times (1-2\%)} + 10\% \\[1.5ex]
		&= 12.24\%
	\end{aligned}
\end{equation*}
\begin{note}
	\begin{itemize}
		\item $D_0$指的是最近已经发放的股利。常见叫法包括“最近支付的股利”、“刚刚支付的每股股利”、“最近一次发放的股利”、“上年的每股股利”、“去年的每股股利”、“今年发放的股利”等;
		\item $D_1$指的是预计要发放的第一期股利。常见的叫法包括“本年将要派发的股利”、“今年将要派发的股利”、“预计第一年的每股股利”、“第一年预期股利”、“预计的本年股利”、“一年后的股利”等。
	\end{itemize}

\end{note}
普通股中的资本资产定价模型可以简要的描述为:普通股投资的必要报酬率等于无风险报酬率加上风险报酬率。用公式表示如下:
\begin{equation*}
	K = R_\mathrm{f} + \beta_i(R_\mathrm{m} - R_\mathrm{f})
\end{equation*}
式中,$K$表示普通股投资的必要报酬率;$R_\mathrm{f}$表示无风险报酬率;$R_\mathrm{m}$表示市场报酬率;$\beta\i$表示第$i$种股票的贝塔系数。

\newpage

\textbf{3. 综合资本成本的计算}

综合资本成本率又称加权平均资本成本率,其计算公式如下:
\begin{equation*}
	K_\omega = \sum_{j=1}^{n} K_jW_j
\end{equation*}
式中,$K_\omega$表示综合资本成本率;$K_j$表示第$j$种长期资本的资本成本率;$W_j$表示第$j$种长期资本的资本比例。并且有:
\begin{equation*}
	\sum_{j=1}^{n} W_j =1
\end{equation*}

在计算综合资本成本时,权重可以有三种选择:账面价值、市场价值和目标价值。账面价值很好理解,市场价值是使用债券和股票等以现行资本市场价格为基础确定其资本比例。目标价值就是按照公司预计未来目标市场价值来确定资本比例。

\textbf{4. 边际资本成本的计算}

\emph{某公司设定的目标资本结构为:银行借款}20\%\emph{、公司债券}15\%\emph{、普通股}65\%\emph{。现拟追加筹资}300\emph{万元,按此资本结构来筹资。个别资本成本率预计为:银行借款}7\%\emph{、公司债券}12\%\emph{、股东权益}15\%\emph{。要求计算追加筹资}300\emph{万的边际资本成本。 }

\solve[答]:边际资本成本的计算通常是以目标价值来确定资本比例的:
\begin{equation*}
	\begin{aligned}
		K_\omega &= 20\% \times 7\% + 15\% \times 12\% + 65\% \times 15\% \\
		&=12.95\%
	\end{aligned}
	
\end{equation*}

\subsection{杠杆效应}

\textbf{1. 营业杠杆系数的计算}

营业杠杆系数反映了企业营业风险的高低,其计算公式为:
\begin{equation*}
	DOL = \frac{\Delta EBIT / EBIT}{\Delta S/ S} =  \frac{\Delta EBIT / EBIT}{\Delta Q/ Q} 
\end{equation*}
式中,$DOL$表示营业杠杆系数;$EBIT$表示营业利润,即息税前利润;$\Delta EBIT$表示营业利润的变动额;$S$表示营业收入;$\Delta S$表示营业收入的变动额;$Q$表示销售数量;$\Delta Q$表示销售数量的变动额。为了方便计算,可以使用下面公式计算:
\begin{equation*}
	DOL = \frac{S-C}{S-C-F} = \frac{Q(P-V)}{Q(P-V)-F}
\end{equation*}
式中,$Q$表示销售数量;$P$表示销售单价;$V$表示单位销售量的变动成本额;$F$表示固定成本总额;$C$表示变动成本总额,可以按变动成本率乘以营业收入总额来确定。一般而言,企业的营业杠杆系数越大,营业杠杆利益和风险就越高。

\textbf{2. 财务杠杆系数的计算}

财务杠杆系数反映了财务杠杆的作用程度,其计算公式为:
\begin{equation*}
	DFL = \frac{\Delta EAT /EAT}{\Delta EBIT / EBIT} = \frac{\Delta EPS/EPS}{\Delta EBIT /EBIT} 
\end{equation*}
式中,$DFL$表示财务杠杆系数;$\Delta EAT$表示税后利润变动额;$EAT$表示税后利润额;$\Delta EPS$表示普通股每股收益变动额;$EPS$表示普通股每股收益额。为了方便计算,可以用下面公式计算:
\begin{equation*}
	DFL = \frac{EBIT}{EBIT-I}
\end{equation*}
式中,$I$表示债务年利息,其余符号含义同前。

\textbf{3. 总杠杆系数的计算}

总杠杆系数是对营业杠杆和财务杠杆的综合程度的量度。其计算公式为:
\begin{equation*}
	DCL = DOL \cdot DFL = \frac{\Delta EPS/EPS}{\Delta Q/Q} = \frac{\Delta EPS/EPS}{\Delta S/S}
\end{equation*}
式中,$DCL$表示总杠杆系数,其他符号含义同前。

\emph{六郎公司年度销售净额为}28000\emph{万元,息税前利润}8000\emph{万元,固定成本为}3200\emph{万元,变动成本率为}60\%;\emph{资本总额为}20000\emph{万元,其中债务资本比例为}40\%\emph{,平均年利润为}8\%\emph{。分别计算三种杠杆系数。}

\solve[答]:营业杠杆系数$DOL$为:
\begin{equation*}
	\begin{aligned}
		DOL = \frac{EBIT+F}{EBIT} 
		= 1+\frac{F}{EBIT} 
		= 1+\frac{3200}{8000} 
		= 1.4
	\end{aligned}
\end{equation*}

财务杠杆系数$DFL$:
\begin{equation*}
	\begin{aligned}
		DFL = \frac{EBIT}{EBIT-I} 
		= \frac{8000}{8000-(20000\times 40\% \times 8\%)} 
		= 1.09
	\end{aligned}
\end{equation*}

总杠杆系数$DCL$:
\begin{equation*}
	\begin{aligned}
		DCL &= DFL \times DOL = 1.09 \times 1.4 
		= 1.53
	\end{aligned}
\end{equation*}
\subsection{资本结构决策}

\textbf{1. 每股收益分析}

每股收益无差别点是指两种或两种以上筹资方案下普通股每股收益相等时的息税前利润点,其计算方法为:
\begin{equation*}
	\frac{(\overline{EBIT}-I_1)(1-T)}{N_1}= \frac{(\overline{EBIT}-I_2)(1-T)}{N_2}
\end{equation*}
式中,$\overline{EBIT}$表示息税前利润平衡点,即每股收益无差别点;$I_1,I_2$表示两种筹资方式下的长期债务年利息;$N_1,N_2$表示两种筹资方式下的普通股股数;$T$表示公司所得税税率。

\newpage

\section{有关第七章的计算}

\subsection{现金流量}

有关营业净现金流量$NCF$和全部现金流量的计算,参考第三章业务计算题的\emph{练习}1。每年营业净现金流量$NCF$的计算公式为:
\begin{equation*}
	\begin{aligned}
		NCF &= \text{年营业收入} - \text{年付现成本} - \text{所得税} \\
	&= \text{税后净利润} + \text{折旧}
	\end{aligned}
\end{equation*}

\subsection{折现现金流量}

\textbf{1. 净现值的计算}

投资项目投入使用后的净现金流量,按资本成本率或企业要求的报酬率折算为现值,减去初始投资以后的余额叫做净现值,其计算公式为:
\begin{equation*}
	NPV = \sum_{t=1}^{n} \frac{NCF_t}{(1+K)^t}-C
\end{equation*}
式中,$NPV$表示净现值;$NCF_t$表示第$t$年的净现金流量;$K$表示折现率(资本成本率);$n$表示项目预计使用年限;$C$表示初始投资额。

计算净现值有以下步骤:
\begin{itemize}
	\item 计算每年的营业净现金流量。
	\item 计算未来现金流量的总现值。
	\item 计算净现值,其公式为:
	\begin{equation*}
		\text{净现值} = \text{未来现金流量的总现值}- \text{初始投资}
	\end{equation*}
\end{itemize}
净现值还可以表述为从投资开始至项目寿命终结时所有的现金流量的现值之和。其一般计算公式为:
\begin{equation*}
	NPV = \sum_{t=0}^{n} \frac{NCF_t}{(1+K)^t}
\end{equation*}

\emph{某公司准备投资一个新的项目以扩充生产能力,预计该项目可以持续}5\emph{年,固定资产投资}750\emph{万元。固定资产采用直线法计提折旧,折旧年限为}5\emph{年,估计净残值为}50\emph{万元。预计每年的付现固定成本为}300\emph{万元,每间产品单价为}250\emph{元,年销量}30000\emph{件,均为现金交易。预计期初需要垫支产品单价为}250\emph{元。假设资本成本率为}10\%\emph{,所得税税率}为25\%\emph{,计算项目营业净现金流量和项目净现值。}

\solve[答]:要计算项目营业净现金流量$NCF$,则必须先计算该项目的折旧额:
\begin{equation*}
	\text{折旧额} = (750-50) \div 5 = 140 \text{万元}
\end{equation*}
计算每年营业净现金流量$NCF$,可以使用现金流量表:
\begin{center}
	\begin{tabular}{ccccccc}
		\toprule
		\emph{项目} & 第0年 & 第1年 & 第2年 & 第3年 & 第4年 & 第5年 \\
		\midrule
		\emph{初始投资} & -750 & & & & & \\
		\emph{残值} & & & & & & 50 \\
		\emph{营运资本} & -250 & & & & & \\
		\emph{营业收入} &  & 750 & 750 & 750 & 750 & 750 \\
		\emph{付现成本} & & 300 & 300 & 300 & 300 & 300 \\
		\emph{折旧} & & 140 & 140 & 140 & 140 & 140 \\
		\emph{税前利润} & & 310 & 310 & 310 & 310 & 310 \\
		\emph{所得税费用} & & 77.5 & 77.5 & 77.5 & 77.5 & 77.5 \\
		\emph{税后净利} & & 232.5 & 232.5 & 232.5 & 232.5 & 232.5 \\
		\emph{项目现金流量} & -1000  & 372.5 & 372.5 & 372.5 & 372.5 & 672.5 \\
		\bottomrule
	\end{tabular}
\end{center}

则净现值$NPV$为:
\begin{equation*}
	\begin{aligned}
		NPV &= \sum_{n}^{t=0} \frac{NCF_t}{(1+K)^t} \\
		&= \frac{-1000}{(1+10\%)^0} + \frac{372.5}{(1+10\%)^1} + \frac{372.5}{(1+10\%)^2} + \frac{372.5}{(1+10\%)^3} \\[1.5ex] &+  \frac{372.5}{(1+10\%)^4} +
		\frac{672.5}{(1+10\%)^5} \\[1.5ex]
		&= 598 \text{万元}
	\end{aligned}
\end{equation*}

\textbf{2. 内含报酬率的计算}

内含报酬率是指使净现值等于零时的折现率,反映投资项目的真实报酬。计算公式为:
\begin{equation*}
	\sum_{t=1}^{n} \frac{NCF_t}{(1+r)^t}-C =0
\end{equation*}
式中,$NCF_t$表示第$t$年的净现金流量;$r$表示内含报酬率;$n$表示项目使用年限;$C$表示初始投资额。

内涵报酬率的计算步骤为(每年的$NCF$相等):
\begin{itemize}
	\item 计算年金现值系数。
	\begin{equation*}
		(P/A,i,n) = \frac{C}{NCF_t}
	\end{equation*}
	\item 查年金系数表,在相同的期数内,找出与上述年金现值系数相邻近的较大和较小的两个折现率。
	\item 根据上述两个邻近的折现率和求得的年金现值系数,采用插值法计算内含报酬率。
\end{itemize}

如果每年的$NCF$不等,则计算比较复杂,在这里不过多赘述。

\textbf{3. 获利指数的计算}

获利指数是投资项目未来报酬的总现值与初始投资额的现值之比。其计算公式为:
\begin{equation*}
	PI = \sum_{t=1}^{n}\frac{NCF_t}{(1+K)^t}/C
\end{equation*}
式中,$PI$表示获利指数,其余符号含义同上,获利指数的计算步骤为:
\begin{itemize}
	\item 计算未来现金流量的总现值。这与计算净现金时采用的方法相同。
	\item 计算获利指数,利用未来现金流量的总现值和初始投资额之比来计算。
\end{itemize}

\textbf{4. 投资回收期的计算}

投资回收期代表收回投资所需的年限,当每年的$NCF$相等时,投资回收期的计算公式为:
\begin{equation*}
	PP = \frac{C}{NCF_t}
\end{equation*}
式中,$PP$表示投资回收期,其余符号含义同上。

\textbf{5. 平均报酬率的计算}

平均报酬率是投资项目寿命周期内平均的年投资报酬率,其计算方法为:
\begin{equation*}
	ARR = \frac{\overline{M}}{C} \times 100\%
\end{equation*}
式中,$ARR$表示平均报酬率;$\overline{M}$表示平均现金流量,其余符号含义同上。

\chapter{模拟の考试}
\section{单项选择题}
\exercise{1} 某公司普通股目前的股价为10元/股,筹资费率为8\%,刚刚支付的每股股利为2元,股利固定增长率3\%,则该企业利用留存收益的资金成本为(\qquad)

A、22.39 \%

B、25.39 \%

C、20.6 \%

D、23.6 \%

\exercise{2} (\qquad)是指在投资收益不确定的情况下,按估计的各种可能收益水平及其发生的概率计算的加权平均数。

A、期望投资收益

B、实际投资收益

C、无风险收益

D、必要投资收益

\exercise{3} 已知国库券利率为5\%,纯利率为4\%,则下列说法正确的是(\qquad)。

A、可以判断目前不存在通货膨胀

B、可以判断目前存在通货膨胀,但是不能判断通货膨胀补偿率的大小

C、无法判断是否存在通货膨胀

D、可以判断目前存在通货膨胀,且通货膨胀补偿率为1\%

\exercise{4} 在现金持有量的成本分析模式和存货模式中均需要考虑的因素包括(\qquad)。

A、管理成本

B、转换成本

C、短缺成本

D、机会成本

\exercise{5} (\qquad)的依据是股利无关论。

A、剩余股利政策

B、固定股利政策

C、固定股利支付率政策

D、低正常股利加额外股利政策

\begin{note}
	\begin{enumerate}
		\item B
		
		留存收益的资本成本的计算和普通股的资本成本的计算相同。题中表明刚刚发放的股利代表的是$D_0$,因此应该首先计算未支付的股利$D_1$:
		\begin{equation*}
			D_1 = 2 \times (1+3\%) = 2.06 \text{元}
		\end{equation*}
		将数据带入普通股的资本成本公式中:
		\begin{equation*}
			\begin{aligned}
				K &= \frac{D_1}{P_0(1-f)} +G \\[1.5ex] 
				&= \frac{2.06}{10 \times (1-8\%)} + 3\% \\[1.5ex]
				&=25.39\%
			\end{aligned}
		\end{equation*}
		\item A
		
		期望投资收益是指在投资收益不确定的情况下,按估计的各种可能收益水平及其发生的概率
		计算的加权平均数。
		\item D
		
		因为国库券风险很小,所以$\text{国库券利率}=\text{无风险收益率纯利率}+\text{通货膨胀补偿率}$,即$5\%=4\%+\text{通货膨胀补偿率}$,因此可以判断存在通货膨胀,且通货膨胀补偿率$=5\%-4\%=1\%$。
		\item D
		
		现金持有量的成本分析模式考虑的是机会成本和短缺成本,现金持有量的存货模式考虑的是机会成本和固定性转换成本,所以二者均考虑的因素是机会成本。
		\item A
		
		 BCD的依据都是股利重要论,只有A是根据股利无关论。股利无关论(也称MM理论)认为,在一定的假设条件限定下,股利政策不会对公司的价值或股票的价格产生任何影响。一个公司的股票价格完全由公司的投资决策的获利能力和风险组合决定,而与公司的利润分配政策无关。
	\end{enumerate}
\end{note}

\newpage

\section{判断题}

\exercise{1} 债权资本和股权资本的权益性质不同,因此两种资本不可以互相转换。

\exercise{2} 当企业实际的现金余额与最佳的现金余额不一致时,可采用短期融资策略或投资于有价证券等策略来达到理想状况。

\exercise{3} 在市场经济条件下,报酬和风险是成反比的,即报酬越大,风险越小。

\exercise{4} 企业的资本金包括实收资本和资本公积金。

\exercise{5} 企业对股权资本依法享有经营权,在企业存续期内,投资者除依法转让外,还可以随时抽回其投入的资本。

\begin{note}
	\begin{enumerate}
		\item 错误
		\item 正确
		\item 错误
		\item 正确
		\item 错误
	\end{enumerate}
\end{note}

\section{名词解释}

\exercise{1} 内含报酬率

\exercise{2} 优先股

\exercise{3} 净现值

\begin{note}
	\begin{enumerate}
		\item \textbf{内涵报酬率}是指能够使未来现金流入量现值等于未来现金流出量现值的折现率,或者说是使投资方案净现值为零的折现率。该指标反映了投资项目的真实报酬。
		\item \textbf{优先股}是公司发行的优先于普通股股东分取股利和公司剩余财产的股票。
		\item \textbf{净现值}是投资项目投入使用后的净现金流量,按资本成本率或企业要求达到的报酬率折算为现值,减去初始投资以后的余额。
	\end{enumerate}
\end{note}

\newpage

\section{简答题}
\exercise{1} 企业筹资组合策略有哪些?

\begin{note}
	\begin{itemize}
		\item 正常的筹资组合:短期资产由短期资金来融通,长期资产由长期资金来融通。
		\item 冒险的筹资组合:将部分长期资产由短期资金来融通。该种策略的资金成本较低,因而能减少利息支出,增加企业收益。但用短期资金融通了一部分长期资产,风险比较大。
		\item 保守的筹资组合:将部分短期资产用长期资金来融通,采用该种筹资方式,短期资产的一部分和全部长期资产都用长期资金来融通,而只有一部分短期资产用短期资金来融通。该种策略的风险较小,但成本较高,会使企业的利润减少。
	\end{itemize}
\end{note}

\section{业务计算题}
\exercise{1} 某公司因业务发展需要,准备购入一套设备。现有甲、乙两个方案可供选择,其中甲方案需投资20万元,使用寿命为5年,采用直线法计提折旧,5年后设备无残值。5年中每年销售收入为8万元每年的付现成为3万,乙方案需投资24万元也也采用直线法计提折旧,使用寿命也为5年,5年后有残值收入4万元。5年中每年的销售收入为10万,付现成本第一年为4万元,以后随着设备不断陈旧将逐年将增加日常修理费2000元,另需垫支营运资金3万元,假设所得税率为40\%。请计算两个方案各年的现金流量分别是多少?
\begin{note}
	计算现金流量,首先必须计算两个方案每年的折旧额。
	
	甲方案每年的折旧额$= 200000 \div 5 = 40000$元。
	
	乙方案每年的折旧额$= (240000 - 40000) \div 5 =40000 $元
	
	由于甲方案没有残值,因此甲方案每年的营业净现金流量$NFC$为:
	\begin{equation*}
		\begin{aligned}
			NCF_1 &= \cdots = NCF_5 \\
			&= \text{税后净利润} + {折旧} \\
			&= (80000-30000-40000)\times (1-40\%) + 40000 \\
			&= 46000 \text{元}
		\end{aligned}
	\end{equation*}
	由于乙方案有残值收入,因此其每年的营业现金流量不相同:
	\begin{equation*}
		\begin{aligned}
			NCF_1 &= (100000-40000-40000) \times (1-40\%) + 40000 = 52000\text{元} \\
			NCF_2 &= (100000-42000-40000) \times (1-40\%) +40000 =50800 \text{元} \\
			NCF_3 &= (100000-44000-40000) \times (1-40\%) +40000 =49600 \text{元} \\
			NCF_4 &= (100000-46000-40000) \times (1-40\%) +40000 =48400 \text{元} \\
			NCF_5 &= (100000-48000-40000) \times (1-40\%) +40000 =47200 \text{元} \\
		\end{aligned}
	\end{equation*}
	根据计算结果编制两个方案的全部现金流量表。
	\begin{center}
		\begin{tabular}{ccccccc}
			\toprule
			项目 & 第0年 & 第1年 & 第2年 & 第3年 & 第4年 & 第5年 \\
			\midrule
			\textbf{甲方案} &  & & & & & \\
			\emph{固定资产投资} & -200000 & & & & & \\
			\emph{营业净现金流量} & & 46000 & 46000 & 46000 & 46000 & 46000\\
			\emph{现金流量合计} & -200000 & 46000 & 46000 & 46000 & 46000 & 46000\\
			\textbf{乙方案}  & & & & & & \\ 
			\emph{固定资产投资} & -240000 & & & & & \\
			\emph{营运资本垫支} & -30000 & & & & & \\
			\emph{营业净现金流量} & & 52000 & 50800 & 49600 & 48400 & 47200\\
			\emph{固定资产残值} &  & & & & & 40000 \\
			\emph{营运资本回收} & & & & & & 30000 \\
			\emph{现金流量合计} & -270000 & 52000 & 50800 & 49600 & 48400 & 117200\\
			\bottomrule
		\end{tabular}
	\end{center}

\end{note}

\exercise{2} 已知国库券的利息率为4\%,证券市场组合的报酬率为12\%,要求:

(1)计算市场风险报酬率。

(2)当$\beta$值为1.5时,必要报酬率应为多少。

(3)如果一投资计划的$\beta$值为0.8,期望报酬率为9.8\%,是否应该进行投资。

(4)如果某股票的必要报酬率为11.2\%,其$\beta$值应为多少。

\begin{note}
	(1)已知国库券的利息率为4\%,则说明无风险报酬率$R_\mathrm{F} = 4\%$。根据公式求出市场风险报酬率$R_\mathrm{P}$为:
	\begin{equation*}
		\begin{aligned}
				R_\mathrm{P} &= R_\mathrm{M} - R_\mathrm{F} =0.12 - 0.04 =8\%
		\end{aligned}
	\end{equation*}
	(2)根据公式求出必要报酬率$R_1$为:
	\begin{equation*}
		\begin{aligned}
			R_1 &= R_\mathrm{F} + \beta_1 (R_\mathrm{M} - R_\mathrm{F}) \\
			&= 0.04 + 1.5 \times (0.12-0.04) = 16\%
		\end{aligned}
	\end{equation*}
	(3)将$\beta_2 =0.8$带入公式得到必要报酬率$R_2$为:
	\begin{equation*}
		\begin{aligned}
			R_2 &= R_\mathrm{F} + \beta_2 (R_\mathrm{M} - R_\mathrm{F}) \\
			&= 0.04 + 0.8 \times (0.12-0.04) = 10.4\% > 9.8\%
		\end{aligned}
	\end{equation*}
	因为求出的必要报酬率大于期望报酬率$R_2 > \overline{R}$,所以不应该进行投资。
	
	(4)已知必要报酬率的公式为:
	\begin{equation*}
		R_i = R_\mathrm{F} + \beta_i (R_\mathrm{M}- R_\mathrm{F})
	\end{equation*}
	将$\beta_i$移项到等式的左边,得到:
	\begin{equation*}
		\beta_i = \frac{R_i-R_\mathrm{F}}{R_\mathrm{M}-R_\mathrm{F}}
	\end{equation*}
	将必要报酬率带入公式得到$\beta$为:
	\begin{equation*}
		\beta_3 = \frac{0.112-0.04}{0.12-0.04} = 0.9
	\end{equation*}
\end{note}
	
\exercise{3} 某企业有甲、乙两个投资项目,计划投资额均为1000万元,其收益率的概率分布如下表所示:

\begin{center}
	\begin{tabular}{cccc}
		\toprule
		市场状况 & 概率 & 甲项目 & 乙项目 \\
		\midrule
		好 & 0.3 & 20\% & 30\% \\
		一般 & 0.5 & 10\% & 10\% \\
		差 & 0.2 & 5\% & -5\% \\ 
		\bottomrule
	\end{tabular}
\end{center}

请回答下列问题:

(1)分别计算甲乙两个项目收益率的期望值。

(2)分别计算甲乙两个项目收益率的标准差。

(3)比较甲乙两个投资项目风险的大小。

\begin{note}
	(1)对于甲项目来说,其期望报酬率$\overline{R}_\mathrm{A}$为:
	\begin{equation*}
		\begin{aligned}
			\overline{R}_\mathrm{A} &= 0.3 \times 20\% + 0.5 \times 10\% + 0.2 \times 5\% \\
			&=  0.12 = 12\%
		\end{aligned}
	\end{equation*}
对于乙项目来说,其期望报酬率$\overline{R}_\mathrm{B}$为:
\begin{equation*}
	\begin{aligned}
		\overline{R}_\mathrm{B} &= 0.3 \times 30\% + 0.5 \times 10\% + 0.2 \times -5\% \\
		&=  0.13 = 13\%
	\end{aligned}
\end{equation*}

(2)对于甲项目来说,其标准差$\sigma_\mathrm{A}$为:
\begin{equation*}
	\begin{aligned}
		\sigma_\mathrm{A} &= \sqrt{(0.2-0.12)^2 \times 0.3 + (0.1-0.12)^2 \times 0.5 + (0.05-0.12)^2 \times 0.2 } \\
		& = 5.57\%
	\end{aligned}
\end{equation*}
对于乙项目来说,其标准差$\sigma_\mathrm{B}$为:
\begin{equation*}
	\begin{aligned}
		\sigma_\mathrm{B} &= \sqrt{(0.3-0.13)^2 \times 0.3 + (0.1-0.13)^2 \times 0.5 + (-0.05-0.13)^2 \times 0.2 } \\
		& = 12.5\%
	\end{aligned}
\end{equation*}

(3)由于甲、乙两个项目的期望值和标准差都不相同,则求其离散系数。甲项目的离散系数$C_\mathrm{A}$为:
\begin{equation*}
	C_\mathrm{A} = \frac{\sigma_\mathrm{A}}{\overline{R}_\mathrm{A}} = \frac{5.57\%}{12\%} = 46.4\%
\end{equation*}
乙项目的离散系数$C_\mathrm{B}$为:
\begin{equation*}
	C_\mathrm{B} = \frac{\sigma_\mathrm{B}}{\overline{R}_\mathrm{B}} = \frac{12.5\%}{13\%} = 96.2\%
\end{equation*}
由于$C_\mathrm{B}>C_\mathrm{A}$,因此投资甲项目的风险较小,投资乙项目的风险较大。
\end{note}

\exercise{4} 某人决定分别在2002年、2003年、2004年和2005年各年的1月1日分别存入5000元,按10\%利率,每年复利一次,要求计算2005年12月31日的余额是多少?(当$i=10\%,n=4$年金终值系数为4.6410,当$i=10\%,n=5$年金终值系数为6.1051)

\begin{note}
	由题意知,年金分别从每年的年初存入,则题目所求余额应为先付年金的本利和。套用先付年金终值公式,则可得到答案:
	\begin{equation*}
		\begin{aligned}
			F &= A \times (F/A,i,n) \times (1+i) \\
			  &= 5000 \times 4.6410 \times 1.1 \\
			  &= 25525.5 \text{元}
		\end{aligned}
	\end{equation*}
	需要注意的是,使用以上方法应该使用$i=10\%,n=4$的年金终值系数4.6410,而如果使用$i=10\%,n=5$的年金终值系数6.1051,则需要套用另外一个公式:
	\begin{equation*}
		\begin{aligned}
			F &= A \times (F/A,i,n+1) -A \\
			  &= 5000 \times 6.1051 - 5000 \\
			  &= 25525.5 \text{元}
		\end{aligned}
	\end{equation*}
\end{note}

\end{document}